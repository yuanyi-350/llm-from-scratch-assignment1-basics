\documentclass[a4paper,12pt]{article}

\usepackage{ctex}
\usepackage{amsmath}
\usepackage{amssymb}
\usepackage{tikz-cd}
\usepackage{geometry}
\geometry{left=2.5cm, right=2.5cm, top=2.5cm, bottom=2.5cm}

\setlength{\parindent}{0pt}

\title{反向传播推导笔记}
\date{\today}

\begin{document}

\maketitle


\paragraph{Affine Forward Pass}

设输入为 $X$, 权重为 $W$, 输出为 $Y$ ($out$)
\begin{itemize}
    \item 输入 $X$: 形状为 $(N, D)$
    \item 权重 $W$: 形状为 $(D, M)$
    \item 输出 $Y = XW$: 形状为 $(N, M)$
    \item 损失 $L$: 标量
\end{itemize}

\paragraph{Affine Backward Pass}

$$
\left(\begin{array}{ll}
x_{11} & x_{12} \\
x_{21} & x_{22} \\
x_{31} & x_{32}
\end{array}\right) \xrightarrow{\left(\begin{array}{ll}
y_{11} & y_{12} \\
y_{21} & y_{21}
\end{array}\right)}\left(\begin{array}{ll}
x_{11} y_{11}+x_{21} y_{21} & x_{11} y_{12}+x_{22} y_{22} \\
x_{21} y_{11}+x_{22} y_{21} & x_{21} y_{12}+x_{22} y_{22} \\
x_{11} y_{11}+x_{32} y_{21} & x_{31} y_{12}+x_{32} y_{22}
\end{array}\right) \longrightarrow L
$$

求 Loss 对输入 $X$ 的梯度 $\frac{\partial L}{\partial X}$, 其形状应与 $X$ 相同, 即 $(N, D)$. 
已知上游梯度 $dout = \frac{\partial L}{\partial Y}$, 形状为 $(N, M)$. 

\subparagraph{微观视角}

考虑任意元素 $X_{ij}$ 对 $L$ 的影响. $X_{ij}$ 参与了 $Y$ 第 $i$ 行所有元素的计算.

\begin{figure}[h]
    \centering
    \begin{tikzcd}
        && {out_{21}} \\
        {x_{21}} &&&& L \\
        && {out_{22}}
        \arrow[from=1-3, to=2-5]
        \arrow[from=2-1, to=1-3]
        \arrow[from=2-1, to=3-3]
        \arrow[from=3-3, to=2-5]
    \end{tikzcd}
    \caption{Affine}
\end{figure}


\begin{itemize}
    \item $\dfrac{\partial L}{\partial Y_{ik}} = (dout)_{ik}$
    \item $\dfrac{\partial Y_{ik}}{\partial X_{ij}} = W_{jk}$
\end{itemize}

根据多元复合函数的链式法则: 
$$
\frac{\partial L}{\partial X_{ij}} = \sum_{k=1}^M \frac{\partial L}{\partial Y_{ik}} \cdot \frac{\partial Y_{ik}}{\partial X_{ij}}
= \sum_{k=1}^M (dout)_{ik} \cdot W_{jk} 
= (dout \cdot W^T)_{ij}
$$

矩阵形式为 $\frac{\partial L}{\partial X} = dout \cdot W^T$ , 维度检查: $(N, M) \times (M, D) = (N, D)$, 与 $X$ 形状一致.

同理可证对权重 $W$ 的梯度 $\frac{\partial L}{\partial W} = X^T \cdot dout$

\subparagraph{理解} 函数 $L$ 是中间变量 $u, v, w$ 的函数 $L=f(u, v, w)$
而 $u, v, w$ 又全都是 $x$ 的函数 $u=g(x), \quad v=h(x), \quad w=k(x)$
则 $L$ 对 $x$ 的导数

$$
\frac{d L}{d x}=\frac{\partial L}{\partial u} \cdot \frac{d u}{d x}+\frac{\partial L}{\partial v} \cdot \frac{d v}{d x}+\frac{\partial L}{\partial w} \cdot \frac{d w}{d x}
$$

这是因为记 $\varphi(x) = (g(x), h(x), k(x))$, 然后 $L = f(\varphi(x))$ 即知

$$
DL(x) = Df(\varphi(x)) \circ D\varphi(x)
$$

也就是

$$
\frac{d L}{d x}=\nabla L(\varphi(x)) \cdot \nabla \varphi(x) 
$$




\paragraph{Layernorm Forward Pass}

\paragraph{Layernorm Backward Pass}

\begin{figure}[h]
    \centering
    \begin{tikzcd}
    & {\text{mean}\;\mu} \\
    x && {\hat{x_i} = \dfrac{x_i - \mu}{\sqrt{\sigma^2 + \varepsilon}}} & L \\
    & {\text{var}\;\sigma^2}
    \arrow[from=1-2, to=2-3]
    \arrow[from=2-1, to=1-2]
    \arrow[from=2-1, to=2-3]
    \arrow[from=2-1, to=3-2]
    \arrow[from=2-3, to=2-4]
    \arrow[from=3-2, to=2-3]
    \end{tikzcd}
    \caption{LayerNorm}
\end{figure}


假设输入 $x$ 的形状为 $(N, D)$ ,其中 $N$ 是样本数,$D$ 是特征维度。Layer Normalization 对每个样本 $x_i$(行向量,长度为 $D$ ) 独立进行归一化。

对于第 $i$ 个样本 $x_i(i \in\{1, \ldots, N\})$ ,其元素为 $x_{i j}(1 \le j \le D)$
\begin{enumerate}
    \item 均值(Mean) $\mu_i=\frac{1}{D} \sum_{j=1}^D x_{i j}$
    \item 方差(Variance) $\sigma_i^2=\frac{1}{D} \sum_{j=1}^D\left(x_{i j}-\mu_i\right)^2$
    \item 标准差(Standard Deviation) $\sigma_i=\sqrt{\sigma_i^2+\epsilon}$
    \item 归一化(Normalize) $\hat{x}_{i j}=\frac{x_{i j}-\mu_i}{\sigma_i}$
\end{enumerate}

计算 Loss $L$ 对输入 $x_{ij}$ 的梯度. 令 $g_{ij} = \frac{\partial L}{\partial \hat{x}_{ij}} = \frac{\partial L}{\partial y_{ij}} \cdot \gamma_j$. 
根据多元微积分的链式法则

$$
\frac{\partial L}{\partial x_{ij}} = 
\underbrace{\frac{\partial L}{\partial \hat{x}_{ij}} \frac{\partial \hat{x}_{ij}}{\partial x_{ij}}}_{\text{Term 1}} + 
\underbrace{\frac{\partial L}{\partial \sigma_i^2} \frac{\partial \sigma_i^2}{\partial x_{ij}}}_{\text{Term 2}} + 
\underbrace{\frac{\partial L}{\partial \mu_i} \frac{\partial \mu_i}{\partial x_{ij}}}_{\text{Term 3}}
$$

\subparagraph{Term 1} 由于 $\hat{x}_{ij} = (x_{ij} - \mu_i)\sigma_i^{-1}$, 固定 $\mu, \sigma$ 时: 
$$
\frac{\partial \hat{x}_{ij}}{\partial x_{ij}} = \frac{1}{\sigma_i} \implies \text{Term 1} = \frac{g_{ij}}{\sigma_i}
$$

\subparagraph{Term 2} 首先计算 $L$ 对方差的总梯度(需要对特征维度 $D$ 进行聚合) : 
$$
\begin{aligned}
\frac{\partial L}{\partial \sigma_i^2} &= \sum_{k=1}^D \frac{\partial L}{\partial \hat{x}_{ik}} \frac{\partial \hat{x}_{ik}}{\partial \sigma_i^2} 
= \sum_{k=1}^D g_{ik} \cdot \left( -\frac{1}{2} \frac{x_{ik} - \mu_i}{(\sigma_i^2 + \epsilon)^{3/2}} \right)
= -\frac{1}{2\sigma_i^2} \sum_{k=1}^D (g_{ik} \hat{x}_{ik}) \\
\frac{\partial \sigma_i^2}{\partial x_{ij}} &= \frac{\partial}{\partial x_{ij}} \left( \frac{1}{D} \sum_{k=1}^D (x_{ik} - \mu_i)^2 \right) = \frac{2(x_{ij} - \mu_i)}{D} = \frac{2 \hat{x}_{ij} \sigma_i}{D}
\end{aligned}
$$

两者相乘得到 Term 2: 
$$
\text{Term 2} = \left( -\frac{1}{2\sigma_i^2} \sum_{k=1}^D g_{ik} \hat{x}_{ik} \right) \cdot \left( \frac{2 \hat{x}_{ij} \sigma_i}{D} \right) 
= -\frac{1}{D \sigma_i} \hat{x}_{ij} \sum_{k=1}^D (g_{ik} \hat{x}_{ik})
$$

\subparagraph{Term 3} 
首先计算 $L$ 对均值的总梯度: 
$$
\frac{\partial L}{\partial \mu_i} = \sum_{k=1}^D g_{ik} \frac{\partial \hat{x}_{ik}}{\partial \mu_i} 
= \sum_{k=1}^D g_{ik} \left( -\frac{1}{\sigma_i} \right) 
= -\frac{1}{\sigma_i} \sum_{k=1}^D g_{ik} , \qquad
\frac{\partial \mu_i}{\partial x_{ij}} = \frac{1}{D}
$$
两者相乘得到 Term 3: 
$$
\text{Term 3} = -\frac{1}{D \sigma_i} \sum_{k=1}^D g_{ik}
$$

\subparagraph{最终}
$$
\frac{\partial L}{\partial x_{ij}} = \frac{1}{D \sigma_i} \left[ D \cdot g_{ij} - \hat{x}_{ij} \left( \sum_{k=1}^D g_{ik} \hat{x}_{ik} \right) - \sum_{k=1}^D g_{ik} \right]
$$

\end{document}